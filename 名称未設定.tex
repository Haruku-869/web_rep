\documentclass[uplatex,dvipdfmx]{jsarticle}

\usepackage[uplatex,deluxe]{otf} % UTF
\usepackage[noalphabet]{pxchfon} % must be after otf package
\usepackage{stix2} %欧文&数式フォント
\usepackage[fleqn,tbtags]{mathtools} % 数式関連 (w/ amsmath)
\usepackage{hira-stix} % ヒラギノフォント&STIX2 フォント代替定義(Warning回避)
\usepackage{float}

\begin{document}

\title{webプログラミング仕様書}
\author{24G1009 井澤晴空}
\date{2025年1月5日}
\maketitle
\section{利用者向け}
今回僕が追加した掲示板は,ユーザーが特定のカテゴリーに投稿をし,他のユーザーの投稿簡単に閲覧できることを目的とした.また,投稿内容をキーワードやカテゴリーで検索する機能を追加することで,利用者が興味のある投稿を機能追加前の掲示板よりも効率的に見つけることができると考える.


\subsection{追加した機能の説明}

\subsubsection{いいね機能} 
ユーザーが投稿した発言に対していいねを押すことができる機能.投稿一覧に表示されている各投稿の一番下にいいねと書いてあるボタンが有りそのボタンを押すことで自分がいいねと思った投稿にいいねをすることができる.

\subsubsection{キーワード検索機能}
投稿した人の名前や発信したメッセージに含まれるキーワードに関連する投稿を表示する機能.自分が気になる話題について話している人を探す際にキーワードで検索と書いてある場所にキーワードを打ち込むことで,キーワード検索の下に投稿者と投稿内容が表示される.
\subsubsection{カテゴリー検索機能}
投稿する際に,日常,勉強,緊急の中から選んだカテゴリーごとに投稿を一覧表示する機能.緊急で話したいことがある場合でも,キーワードを緊急に設定することで,そのキーワードで検索して見てくれている人にすぐに見てもらえるようにした.

\subsection{操作方法}
\begin{enumerate}
\item 投稿サイトに名前,メッセージ,カテゴリーを入力する.
\item 投稿するボタンをクリックして投稿を送信する.
\item カテゴリーボタンまたはキーワード入力欄にキーワードを入力して,関連する投稿を検索する.
\end{enumerate}
\section{管理者向け}
\subsection{サーバーの起動}
以下のコマンドでサーバーを起動する.
\begin{verbatim}
node server.js
\end{verbatim}
これにより,\texttt{http://localhost:3000} でサーバーにアクセスできるようになる.
\section{システムの動作確認}
\subsection{投稿機能の確認}
利用者がメッセージを投稿できることを確認する.投稿フォームに名前,メッセージ,キーワードを入力し,投稿ボタンをクリックする.正常に投稿され,投稿されたメッセージが投稿一覧に表示されることを確認する.
\subsection{いいね機能の確認}
投稿されたメッセージにいいねボタンが表示されていることを確認し,ボタンをクリックし,メッセージのいいねの数が増加することを確認する.
\subsection{キーワード検索機能の確認}
キーワード検索機能が正しく動作することを確認する.検索ボックスにキーワードを入力し,検索ボタンをクリックする.検索結果として,適切なメッセージが表示されることを確認する.
\subsection{カテゴリー検索機能の確認}
カテゴリー検索機能が正しく動作することを確認する.カテゴリーのボタンを押し,押したカテゴリーの投稿一覧が表示されることを確認する.

\section{開発者向け}
\section{内部データについて}
\subsection{いいね機能}
各投稿にはいいねボタンがあり,利用者がいいねボタンクリックすることで,その投稿に対するいいね数が増加する.情報は,サーバーにPOSTリクエストとして送信され,いいね数が返される.
\subsection{カテゴリー別表示}
投稿は日常,勉強,緊急の3つのカテゴリーごとに分類され,利用者がカテゴリを選択すると,そのカテゴリに関連する投稿が表示される.このリクエストは,カテゴリ名をPOSTリクエストとしてサーバーに送信し、該当する投稿を返す.
\subsection{キーワード検索}
利用者がキーワードを入力して検索すると,関連する投稿が表示される.キーワードはサーバーに送信され,そのキーワードに関連する投稿が返されれる.

\subsection{通信フロー}
各機能の通信はPOSTメソッドを使用して行っている.
データはすべてJSON形式のためクライアントとサーバー間で高速に処理される.

\section{Github}
\end{document} 
